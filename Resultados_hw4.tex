\documentclass[12pt,letterpaper]{article}
\usepackage[utf8]{inputenc}
\usepackage[spanish]{babel}
\usepackage{graphicx}
\usepackage{float}
\usepackage{enumerate} 

\begin{document}
\begin{center}
{\textbf{Tarea 4 - A. Lozada. 201425109}}\\
\vspace{0.2cm}
\end{center}


En el enunciado de la tarea sugerían un coeficiente de difusión de valor $\nu= 10^{-4}$. Yo, después de algunas consideraciones, escogí uno distinto (igual a $0.1$) pues reorganicé la unidades para que esta fuera la medida. 

\section{F1}
\vspace{0.2cm}

\begin{center}
{\textbf{Condiciones de frontera fijas para el caso 1 en los tres tiempos.}}\\
\vspace{0.2cm}
\end{center}


\begin{figure}[H]
\includegraphics{1F0.png}
\centering
\end{figure}

\begin{figure}[H]
\includegraphics{1F100.png}
\centering
\end{figure}

\begin{figure}[H]
\includegraphics{1F2500.png}
\centering
\end{figure}


\section{A1}
\vspace{0.2cm}

\begin{center}
{\textbf{Condiciones de frontera abiertas para el caso 1 en los tres tiempos.}}\\
\vspace{0.2cm}
\end{center}

\begin{figure}[H]
\includegraphics{1A0.png}
\centering
\end{figure}

\begin{figure}[H]
\includegraphics{1A100.png}
\centering
\end{figure}

\begin{figure}[H]
\includegraphics{1A2500.png}
\centering
\end{figure}



\section{P1}
\vspace{0.2cm}

\begin{center}
{\textbf{Condiciones de frontera periódicas para el caso 1 en los tres tiempos.}}\\
\vspace{0.2cm}
\end{center}

\begin{figure}[H]
\includegraphics{1P0.png}
\centering
\end{figure}

\begin{figure}[H]
\includegraphics{1P100.png}
\centering
\end{figure}

\begin{figure}[H]
\includegraphics{1P2500.png}
\centering
\end{figure}



\section{F2}
\vspace{0.2cm}

\begin{center}
{\textbf{Condiciones de frontera fijas para el caso 2 en los tres tiempos.}}\\
\vspace{0.2cm}
\end{center}


\begin{figure}[H]
\includegraphics{2F0.png}
\centering
\end{figure}

\begin{figure}[H]
\includegraphics{2F100.png}
\centering
\end{figure}

\begin{figure}[H]
\includegraphics{2F2500.png}
\centering
\end{figure}


\section{A2}
\vspace{0.2cm}

\begin{center}
{\textbf{Condiciones de frontera abiertas para el caso 2 en los tres tiempos.}}\\
\vspace{0.2cm}
\end{center}

\begin{figure}[H]
\includegraphics{2A0.png}
\centering
\end{figure}

\begin{figure}[H]
\includegraphics{2A100.png}
\centering
\end{figure}

\begin{figure}[H]
\includegraphics{2A2500.png}
\centering
\end{figure}



\section{P2}
\vspace{0.2cm}

\begin{center}
{\textbf{Condiciones de frontera periódicas para el caso 2 en los tres tiempos.}}\\
\vspace{0.2cm}
\end{center}

\begin{figure}[H]
\includegraphics{2P0.png}
\centering
\end{figure}

\begin{figure}[H]
\includegraphics{2P100.png}
\centering
\end{figure}

\begin{figure}[H]
\includegraphics{2P2500.png}
\centering
\end{figure}



\section{Promedios}

Los promedios funcionan como deberían funcionar. Sin embargo, existe un problema con las condiciones abiertas en el cual me rendí. Es el siguiente: cuando el flujo no es constante, la placa se enfría como debería y lo hace así de rápido porque aumenté el coeficiente de difusión. Ahora, cuando el flujo es constante, por más que en el código ponga que el rectángulo se quede en cien grados, simplemente no lo hace y se enfría. No supe por qué. 

\subsection{Caso 1}

\begin{figure}[H]
\includegraphics{MEAN1.png}
\centering
\end{figure}

\subsection{Caso 2}

\begin{figure}[H]
\includegraphics{MEAN2.png}
\centering
\end{figure}


\vspace{0.3cm}


\end{document}
